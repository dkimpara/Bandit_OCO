\begin{algorithm}
	\begin{algorithmic}
		\caption{\gscore$(\dtriple)$} \label{alg:gscore}
		\State \textbf{Input:} A Dodgson triple $\dtriple$.
		\State
		\For{$d \in C \setminus \{c\}$} \Comment Initialize counter variables
			\State $Deficit[d] \leftarrow 0$ \Comment Number of votes by which $c < d$ ($d$ beats $c$)
			\State $Swaps[d] \leftarrow 0$ \Comment Number of votes by which $c \prec d$ (greedily swappable votes against $d$)
		\EndFor

		\For{each vote $v \in V$} \Comment each vote is an 
		array where $v[k] \prec v[k+1]$
			\State $i \leftarrow 1$
				\While{$v[i] \neq c$}
					\State $d \leftarrow v[i]$
					\State $Deficit[d] \leftarrow Deficit[d]-1$
					\State $i \leftarrow i+1$
				\EndWhile
				\If{$ i < length(v)$}
					\State $d \leftarrow v[i+1]$
					\State $Swaps[d] \leftarrow Swaps[d]+1$
				\EndIf
				\For{$i \leftarrow i + 1$ to $length(v)$ }
					\State $d \leftarrow v[i]$
					\State $Deficit[d] \leftarrow Deficit[d]+1$


				\EndFor
		\EndFor
		\State $confidence \leftarrow$ ``definitely''
		\State $score \leftarrow 0$
		\For{$d \in C \setminus \{c\}$} \Comment If there are more than $1+Deficit[d]/2$ greedily swappable votes,
			\If{$Deficit[d]\geq 0$} \Comment then this is the Dodgson Score of $c$.
				\State $score \leftarrow score +
				\floor{Deficit[d]/2} + 1$ 
				\If{$Deficit[d] \geq 2 \cdot Swaps[d]$}
					\State $confidence \leftarrow$ ``maybe''
					\State $score \leftarrow score + 1$
				\EndIf
			\EndIf
		\EndFor
		\State \textbf{Output:} $(score,~confidence)$
	\end{algorithmic}
\end{algorithm}
\begin{algorithm}
\begin{algorithmic}
	\caption{\gwin$(\dtriple)$} \label{alg:gwin}
	\State \textbf{Input:} A Dodgson triple $\dtriple$ where
	we want to test whether $c$ is a Dodgson winner in the
	election.
	\State $(cscore,~confidence) \leftarrow \gscore(\dtriple)$
	\State $winner \leftarrow $ ``yes''
	\For{$d \in C \setminus\{c\}$}
		\State $(dscore,~confidence) \leftarrow
		\gscore(\langle C,d,V \rangle)$
		\If{$dscore < cscore$}
			\State $winner \leftarrow$ ``no''
		\If{$dcon = $ ``maybe''}
			\State $confidence \leftarrow$ ``maybe''
		\EndIf
		\EndIf
	\EndFor
	\State \textbf{Output:} $(winner,~confidence)$
\end{algorithmic}
\end{algorithm}
\section{$d+1$ point feedback}
In this section, we show that we can construct a deterministic gradient estimator using $d+1$ point feedback. Thus we obtain a deterministic version of Theorem \ref{thm:one}. Hence, the algorithm is no-regret even against completely adaptive adversaries meaning that the adversary can choose the loss $\lt$ after the algorithm plays $x_t$. Hence we match the full-information bound.

The algorithm constructs the deterministic gradient estimator
$$\tildegt = \frac{1}{\delta} \sum_{i=1}^{d} (\lt(x_t + \delta e_i)-\lt(x_t))e_i.$$

Where $e_i$'s are the standard unit basis vectors. We further need only the assumptions on strong convexity and $L$-smoothness, since they imply a bound on the gradient which we denote $G$. We can thus derive a bound on the norm of the estimator

\begin{align*}
	\norm{\tildegt} &= \norm{\frac{1}{\delta} \sum_{i=1}^{d} (\lt(x_t + \delta e_i)-\lt(x_t))e_i} \\
	& \leq \frac{d}{\delta} \max_i \norm{\lt(x_t + \delta e_i)-\lt(x_t)} \\
	& \leq \frac{d}{\delta} \delta G \\
	&= dG.
\end{align*}

Where the second inequality is by the Lipschitz property. We can also derive the divergence of the estimator:

\begin{align*}
	\norm{\tildegt - \glt(x_t)} &= \sqrt{|\frac{1}{\delta} \sum_{i=1}^{d} (\lt(x_t + \delta e_i)-\lt(x_t))e_i - \langle \glt(x_t), e_i \rangle|^2} \\
	& \leq \sqrt{\frac{d}{\delta} \max_i\{|(\lt(x_t + \delta e_i)-\lt(x_t))e_i - \langle \glt(x_t), e_i \rangle|^2\}} 
\end{align*}

By the smoothness assumption, we have for all $i$

$$\lt(x_t + \delta e_i) \leq \lt(x_t) + \delta \langle \glt(x_t), e_i \rangle + \frac{L\delta^2}{2}.$$

And by convexity we have $\lt(x_t + \delta e_i) \geq \lt(x_t) + \delta \langle \glt(x_t), e_i \rangle$. Hence 

$$|\frac{1}{\delta}(\lt(x_t + \delta e_i)-\lt(x_t))e_i - \langle \glt(x_t), e_i \rangle|^2 \leq \frac{L^2\delta^2}{4}.$$

So we conclude that 

$$\norm{\tildegt - \glt(x_t)} \leq \frac{\sqrt{d}L\delta}{2}$$


Hence we have that the properties of $\tildegt$ are the deterministic version of the properties of the estimator outline in theorem \ref{thm:one}. Hence we have an algorithm that can guarantee no-regret against a completely adaptive adversary. 


\begin{theorem} \label{thm:dplusone}
	Suppose a completely adaptive adversary chooses the sequence of loss functions $\{\lt\}_{t=1}^T$ subject to the same assumptions as above. If Algorithm \ref{alg} is run with the $\eta \leq \frac{1}{2dG}$, $\delta = \frac{\log(T)}{T}$, and $\xi = \frac{\delta}{r}$, then
	
	$$\sumt \frac{1}{d+1} (\lt(x_t) + \sum_{i=1}^{d}\lt(x_t+\delta e_i)) - \sumt \lt(\xst) \leq
	Regret_T^d(OGD, dG) + G \log(T) \big(1 + \frac{\sqrt{d} L \delta}{2}
	+ \frac{D}{r}\big).$$
\end{theorem}

\begin{proof}
This is a modification of the proof of Theorem \ref{thm:one}. Define $h_t(x) = \lt(x) + (\tildegt - \glt(x))^\intercal x$. Then since $\norm{\nabla h_t(x_t)} \leq dG$. Hence from Lemma \ref{lem:OGD} for any sequence $\{\xst\}_{t=1}^T$,

$$\sumt h_t(x_t) - h_t(\xst) \leq Regret_T^d(OGD, dG).$$

Then we proceed as in the proof of Theorem \ref{thm:one}, use $\norm{\tildegt - \glt(x_t)} \leq \frac{\sqrt{d} L \delta}{2}$. Apply Lemma \ref{lem:2} and plug in our parameters and this gives us the result.

\end{proof}

