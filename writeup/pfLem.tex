\subsection{Proof of select lemmas} \label{subsec:pflem}

%\begin{problem}{\algprobm{Comp-3DM}}
%    \tab \textbf{Input:}
%    \tab \textbf{Decide:}
%\end{problem}

\begin{pfsketch}{Lemma \ref{lem:red1} or
                    \algprobm{3DM} $\mred$ \algprobm{DodgsonScore}}\\
    This reduction differs from \citet{bartholdiVoting}
    in that this reduction has additional properties that are required
    by the lemma.
    We will reduce from \algprobm{ThreeDimensionalMatching} to \dscore.
    This reduction has many technical details so we provide only a
    very brief sketch.

\begin{problem}{\algprobm{ThreeDimensionalMatching} (\algprobm{3DM})}\\
    \tab \textbf{Input:} Sets $M,W,X,Y$, where $M\subseteq W \times
    X \times Y$ and $W,X,Y$ are disjoint, nonempty sets having the
    same number of elements.\\
    \tab \textbf{Decide:} Does $M$ contain a matching, i.e. a subset
    $M' \subseteq M$ such that $\norm{M'} = \norm{W}$
    and no two elements of $M'$ agree in any coordinate?
\end{problem}

    Now given an instance of \algprobm{3DM} as outlined above,
    let $C = W \cup X \cup Y \cup \{c,s,t\}$ where $c,s,t \notin W \cup X \cup Y$.
    Let $V$ consist of voters simulating elements of $M$ and $\norm{M}-1$ dummy
    voters.
    Then, $M$ containing a matching corresponds to $Score(\dtriple) = 3q$ and
    $M$ not containing a matching corresponds to $Score(\dtriple) = 3q+1$.
    $\square$
\end{pfsketch}

\begin{proof}{Lemma \ref{lem:2er}}\\
    We will reduce from \csat to \algprobm{2ER}.
    Let $\langle A,B \rangle$ be a \csat instance.

    For each 3CNF formula $x \in A$ or $B$, reduce $x$ into
    the corresponding \algprobm{3DM} instance $x'$ and add $x'$
    to $A'$ or $B'$ if $x\in A$ or $x \in B$, respectively.
    In effect we are reducing \csat to an instance of
    \algprobm{Comp-3DM}, $\langle A', B' \rangle$.
    It is easy to see that solving $\langle A', B' \rangle$
    solves $\langle A, B \rangle$.
    One can also see that \algprobm{Comp-3DM} is \tp-complete
    because it shares the structure where two lists of \np-hard
    problems are compared.

    Now we perform a similar reduction from \algprobm{Comp-3DM}
    to \algprobm{Comp-DodgsonScore}.
    For each $x' \in A'$ or $B'$, use the function in
    Lemma \ref{lem:red1} to reduce $x'$ into the corresponding
    \dscore~instance $\dtriple$ and add $\dtriple$
    to $A''$ or $B''$ if $x'\in A'$ or $x' \in B'$, respectively.
    It is similarly easy to see that solving $\langle A'', B'' \rangle$
    solves $\langle A', B' \rangle$.
    If $x'$ is a yes-instance of \algprobm{3DM} then by
    Lemma \ref{lem:red1},
    \[Score(f(x')) = Score(x'') = Score(\dtriple) = k\]
    where $f$ is the function described in Lemma \ref{lem:red1}.
    Thus $\dtriple$ is also
    a yes-instance of \dscore.
    If $x'$ is a no-instance of \algprobm{3DM} then
    $Score(\dtriple) = k+1$ and the corresponding triple
    $\dtriple$ is a no-instance of \dscore.

    Now we reduce from \algprobm{Comp-DodgsonScore} to \algprobm{2ER}
    using Lemma \ref{lem:merge1}.
    First note that the direction of the inequality of the decision problem
    changes in this reduction by the nature of Lemma \ref{lem:red1}.
    Now to begin the reduction, we merge all the Dodgson elections
    in $A''$ and $B''$ into $\langle C,c,V \rangle$ and $\langle D,d,W \rangle$,
    respectively.
    This is done using the $DodgsonSum$ function in Lemma \ref{lem:merge1},
    which we can use because Lemma \ref{lem:red1} ensures that conditions of
    the Lemma are met by each election.

    For example if $A'' = \dtripnum{1},\dtripnum{2}, \dots, \dtripnum{k}$
    then

    \[\dtriple = DodgsonSum(\langle \dtripnum{1},\dtripnum{2}, \dots,
    \dtripnum{k}) \rangle)\]

    and

    \[\sum_j Score(\dtripnum{j}) =
    Score(\dtriple).\]

    Now $\langle A'', B'' \rangle$ is a yes-instance of
    \algprobm{Comp-DodgsonScore} if by definition,
    the number of satisfied \dscore~instances in $A''$ being greater than
    that of $B''$.
    This again is equivalent to the sum of the Dodgson
    scores of $A''$ being less than that of $B''$ since we fix
    $k$ to be the same for each of the reductions using Lemma \ref{lem:red1}.
    Let $\norm{A}_{yes}$ be the number of yes-instances in a set of decision
    problems $A$.
    Using the reduction of $\langle A'', B'' \rangle$ outlined above,

    \begin{align}
        &\langle A'', B'' \rangle \text{ is a yes-instance of
    \algprobm{Comp-DodgsonScore}}\nonumber\\
        &\iff \norm{A''}_{yes} \geq \norm{B''}_{yes} \nonumber \\
        &\iff \norm{\{x\in A'' | Score(x) \leq k\}} \geq
              \norm{\{x\in B'' | Score(x) \leq k\}} \nonumber\\
        &\iff Score(\langle C,c,V \rangle)\leq
        Score(\langle D,d,W \rangle) \label{eqRED} \\
        &\iff \langle \langle C,c,V \rangle, \langle D,d,W \rangle \rangle
        \text{ is a yes-instance of \algprobm{2ER}} \nonumber
    \end{align}

    For line (\ref{eqRED}), note that by the reduction used from
    Lemma \ref{lem:red1}, the elections in
    $\langle A'',B'' \rangle$ can have score of either $k$ or $k+1$.
    Hence we have shown a reduction from \algprobm{Comp-DodgsonScore} to
    \algprobm{2ER}.

    Combining the many-one reductions above, we have shown that:

    \[\text{\csat} \mred \text{\algprobm{Comp-3DM}} \mred
    \text{\algprobm{Comp-DodgsonScore}} \mred
    \text{\algprobm{2ER}}.\]

    So by Theorem \ref{thm:csat}, \algprobm{2ER} is \tp-hard and the
    Lemma is proved.
\end{proof}

\subsection{Proof that \algprobm{DodgsonWinner} is \tp-complete.}
\begin{proof}{Theorem \ref{thm:dwin}}\\
    By Theorem \ref{thm:intp}, \dwin~$\in$ \tp.
    We now show that \algprobm{2ER} $\mred$ \dwin~and so
    by Lemma \ref{lem:2er}, the theorem then follows.

    We now describe a polynomial time function $f$ for this reduction.
    Let $s_o$ be some fixed string that is not in \dwin.
    Then

    \[
      f(x) =
      \begin{cases}
          Merge(x_1,x_2) & \text{if $x \in$ \algprobm{2ER}}  \\
          s_o & \text{if $x \notin$ \algprobm{2ER}}
      \end{cases}
    \]

    Where $Merge$ is the function defined in Lemma \ref{lem:merge2} and
    $x_1$ and $x_2$ are the two elections in the instance of \algprobm{2ER}.
    So $f(x)$ is an instance of \dwin~if and only if $x$ is an instance of
    \algprobm{2ER}.

    Now we show how $Merge(x_1,x_2)$, an instance of \dwin, solves the
    corresponding instance $x$ of \algprobm{2ER}.
    Let $x$ be a pair of Dodgson triples, $\dtriple$ and
    $\langle D,d,w \rangle$, where both have an odd number of voters
    and $d \neq c$ (so we can apply the Lemma).
    Then let $Merge(\langle C,c,V \rangle, \langle D,d,W \rangle) =
    \htrip{c}$ be the corresponding instance of \dwin.

    First assume that $Score(\langle C,c,V \rangle)\leq Score(\langle D,d,W \rangle)$, or the
    answer to the \algprobm{2ER} decision problem is yes.
    Then by Lemma \ref{lem:merge2}, properties 3 and 4,
    $Score(\htrip{c}) \leq Score(\htrip{d})$.
    By property 5, $Score(\htrip{c}) \leq Score(\htrip{e})$ for all
    $e \in \hat{C}\setminus \{c,d\}$.
    Hence $c$ is a Dodgson winner of the election and the answer
    to the decision problem of this instance of \dwin~is yes.

    Finally, assume that $Score(\langle C,c,V \rangle)> Score(\langle D,d,W \rangle)$, or the
    answer to the \algprobm{2ER} decision problem is no.
    Then similarly, $Score(\htrip{c}) > Score(\htrip{d})$ and
    so $c$ is not a Dodgson winner of the election.
    Hence the answer to the decision problem of this instance of \dwin~is
    no.

\end{proof}