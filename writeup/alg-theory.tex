\section{Practical Greedy Algorithm}\label{subsec:greedy}
In this section we present the \gwin~algorithm by \citet{heuristic}
which runs in polynomial time and is self-knowingly
correct with high probability when the number of
voters is superquadratic in the number of candidates.
This probability is over a uniform random draw of each
vote i.e. $m!$ possibilities for each vote.

In real-world settings, this algorithm seems adequate for most elections.
It is hard to imagine a large election where the number of candidates is
larger than the square root of the number of voters.
For instance theoretically
this algorithm can handle a small town of 2500 with at most about
50 candidates.
We will see empirically that the number is more like 8000 voters for a set
of 50 candidates.
In the real world though, as the number of voters increases
the number of candidates does not seem to grow like $\sqrt{n}$ but more so
some function much slower than that.

While hard problems may not admit efficient deterministic
algorithms, we can still try to create heuristic
algorithms which are correct in many cases.
For example, modern \algprobm{SAT} solvers utilize are a backtracking
algorithm
at their core while using
several heuristics such as variable selection, clause learning,
and conflict directed backjumping to improve their performance.

Theoretically however, there are results that show that
no \np-hard problem has a deterministic heuristic algorithm
whose asymptotic error rate is subexponential \citep{heuristic2012}.
But an algorithm with a larger than subexponential error rate can
still be useful.
For this particular algorithm, \gscore, asymptotically, the algorithm succeeds with
high probability when the number of candidates, $m$, goes to infinity
and the number of voters is super-quadratic in $m$.

Success is when the algorithm is self-knowingly correct.
Intuitively, this means that algorithm will output the value computed along with
a certificate saying that the output is either definitely (provably) correct or ``maybe''
correct.
This is opposed to other algorithms that will output an answer without such
a certificate of correctness and instead have an overall probability of correctness.
We first formally define what we mean by self-knowingly correct.

\begin{defn}
	For sets $S$ and $T$ and function $f: S \rightarrow T$,
	an algorithm $\mathcal{A}: S \rightarrow T \times
	\{\text{``definitely''}, \text{``maybe''}\}$ is
	self-knowingly correct for all $f$ if, for all $s \in S$
	and $t \in T$ whenever $\mathcal{A}$ on input $s$ outputs
	$(t, \text{``definitely''})$ it holds that $f(s) = t$.
\end{defn}



We now explain how the algorithm works.
See below for the full pseudocode for the algorithm.
Our goal is to determine if a particular candidate $c \in C$ is
a Dodgson winner of the election.
What \gwin~does is query \gscore~for the scores of every candidate
and outputs ``definitely'' or ``maybe'' and if $c$ is a Dodgson winner or not.
\gwin~will output ``definitely'' only if all the \gscore~queries output
``definitely''.

Now, what \gscore~does is to go through each vote $v \in V$ and looks at
which candidate is immediately preferred to $c$ (which $d\in C$ such that
$c \prec d$).
If $c$ is not yet beating $d$ in terms of how many voters prefer one candidate
to the other, then \gscore~exchanges $c$ and $d$ in that vote $v$.
Now if $c$ is a Condorcet winner once \gscore~has gone through all the votes,
then the algorithm outputs ``definitely'' and $Score(c)$.

Intuitively, a lower bound for the number of exchanges to make $c$ a Condorcet
winner is the minimum number votes in which a single adjacent exchange takes place
to make $c$ a winner.
If the algorithm does all these exchanges and $c$ is a Condorcet winner then
it is clear that the algorithm does not perform any more exchanges than the
lower bound.
The algorithm thus has computed the Dodgson score of $c$.

\begin{theorem} ~
	\begin{enumerate}
		\item \gscore~is self-knowingly correct for $Score$.
		\item \gwin~is self-knowingly correct for \dwin.
		\item \gscore~and \gwin~both run in polynomial time.
	\end{enumerate}
\end{theorem}

We now explain the probability of success of \gscore~and thus \gwin.
As mentioned above, the \gwin~succeeds with high probability when
$\norm{V} = n$ is superquadratic in $\norm{C} = m$.
This is because when the $n$ is large compared to $m$,
it is more likely that \gscore~will have enough immediately adjacent swap votes
to make up for the deficit between the candidate in question, $c$, and all
the other candidates $d$.
More precisely, for any two candidates $c$ and $d$, in half the ways voter
$v$ can vote, $c <_v d$.
But, $c \prec_v d$ in only $1/m$ of the ways.
Drawing the votes uniformly at random, the number of votes where $c <_v d$ is
$n/2$ and the number of votes where $c \prec_v d$  is $n/m$.
Thus the probability of success is high when $n$ is large compared to $m$.
The probability of success for \gwin~then utilizes the union bound and Chernoff's
Theorem.
\begin{algorithm}
	\begin{algorithmic}
		\caption{\gscore$(\dtriple)$} \label{alg:gscore}
		\State \textbf{Input:} A Dodgson triple $\dtriple$.
		\State
		\For{$d \in C \setminus \{c\}$} \Comment Initialize counter variables
			\State $Deficit[d] \leftarrow 0$ \Comment Number of votes by which $c < d$ ($d$ beats $c$)
			\State $Swaps[d] \leftarrow 0$ \Comment Number of votes by which $c \prec d$ (greedily swappable votes against $d$)
		\EndFor

		\For{each vote $v \in V$} \Comment each vote is an 
		array where $v[k] \prec v[k+1]$
			\State $i \leftarrow 1$
				\While{$v[i] \neq c$}
					\State $d \leftarrow v[i]$
					\State $Deficit[d] \leftarrow Deficit[d]-1$
					\State $i \leftarrow i+1$
				\EndWhile
				\If{$ i < length(v)$}
					\State $d \leftarrow v[i+1]$
					\State $Swaps[d] \leftarrow Swaps[d]+1$
				\EndIf
				\For{$i \leftarrow i + 1$ to $length(v)$ }
					\State $d \leftarrow v[i]$
					\State $Deficit[d] \leftarrow Deficit[d]+1$


				\EndFor
		\EndFor
		\State $confidence \leftarrow$ ``definitely''
		\State $score \leftarrow 0$
		\For{$d \in C \setminus \{c\}$} \Comment If there are more than $1+Deficit[d]/2$ greedily swappable votes,
			\If{$Deficit[d]\geq 0$} \Comment then this is the Dodgson Score of $c$.
				\State $score \leftarrow score +
				\floor{Deficit[d]/2} + 1$ 
				\If{$Deficit[d] \geq 2 \cdot Swaps[d]$}
					\State $confidence \leftarrow$ ``maybe''
					\State $score \leftarrow score + 1$
				\EndIf
			\EndIf
		\EndFor
		\State \textbf{Output:} $(score,~confidence)$
	\end{algorithmic}
\end{algorithm}
\begin{algorithm}
\begin{algorithmic}
	\caption{\gwin$(\dtriple)$} \label{alg:gwin}
	\State \textbf{Input:} A Dodgson triple $\dtriple$ where
	we want to test whether $c$ is a Dodgson winner in the
	election.
	\State $(cscore,~confidence) \leftarrow \gscore(\dtriple)$
	\State $winner \leftarrow $ ``yes''
	\For{$d \in C \setminus\{c\}$}
		\State $(dscore,~confidence) \leftarrow
		\gscore(\langle C,d,V \rangle)$
		\If{$dscore < cscore$}
			\State $winner \leftarrow$ ``no''
		\If{$dcon = $ ``maybe''}
			\State $confidence \leftarrow$ ``maybe''
		\EndIf
		\EndIf
	\EndFor
	\State \textbf{Output:} $(winner,~confidence)$
\end{algorithmic}
\end{algorithm}
\newpage
\begin{theorem}
	For each $m,n \in \mathbb{N}^+$, the following hold.
	Let $C = \{1, \dots, m\}, \norm{V} = n$.
	\begin{enumerate}
		\item For each $c \in C$,
		\[\Pr[\gscore(\dtriple) \neq
		(Score(\dtriple), \text{``definitely''})] < 2(m-1)
		e^{\frac{-n}{8m^2}},\]
		where the probability is taken over
		drawing uniformly at random an $m$-candidate, $n$-voter
		Dodgson election (i.e. all $(m!)^n$ Dodgson elections having
		$m$ candidates and $n$ voters are equally likely to be chosen).
		\item \[\Pr[ \exists c \in C | \gwin(\dtriple) \neq
		(\dwin(\dtriple), \text{``definitely''})] <
		2(m^2-m) e^{\frac{-n}{8m^2}},\]
		where the probability is taken over
		drawing uniformly at random an $m$-candidate, $n$-voter
		Dodgson election.
	\end{enumerate}
\end{theorem}

\begin{proof}
	First we need to establish under what conditions \gscore~is
	self-knowingly correct.
	\begin{claim}
		For each $c \in C$, if for all $d \in C \setminus \{c\}$ it holds
	that

	\begin{align}
	    \norm{\{i \in [n] | c <_{v_i} d\}} &\leq \frac{2mn+n}{4m}, \text{ and}
		\label{c11}\\
	    \norm{\{i \in [n] | c \prec_{v_i} d\}} &\geq \frac{3n}{4m}
		\label{c12}
	\end{align}
	then $\gscore(\dtriple) = (Score(\dtriple), \text{``definitely''})$.

	\end{claim}
	\begin{claimproof}
		$\frac{2mn+n}{4m} = \frac{n}{2} + \frac{n}{4m}$ so if (\ref{c11})
		holds then either $c$ beats $d$ and we are done. Otherwise,
		$d$ beats $c$ and flipping
		$\frac{n}{4m}$ votes where $c \prec d$ to $d \prec c$
		would ensure that $c$ beats $d$.
		If (\ref{c12}) holds then there are strictly more than
		$\frac{n}{4m}$ flippable votes so \gscore~will be able to make enough
		swaps to ensure that $c$ beats $d$.
	\end{claimproof}

	Now we need to show what is the probability that these conditions
	are not met given a possible pair $c,d \in C, c\neq d$.
	We will later use the union bound to present the probability of failure of
	the algorithm overall.
	\begin{claim}
		For each $c,d \in C$ such that $ c\neq d$,
		\[\Pr[(\norm{\{i \in [n] | c <_{v_i} d\}} >
		\frac{2mn+n}{4m}) \vee
		(\norm{\{i \in [n] | c \prec_{v_i} d\}} < \frac{3n}{4m})] <
		2 e^{\frac{-n}{8m^2}}.\]

		Where the probability is taken over
		drawing uniformly at random an $m$-candidate, $n$-voter
		Dodgson election.
	\end{claim}
	\begin{pfsketch}
		Use the Chernoff theorem to show that
		\begin{align}
		\Pr[\norm{\{i \in [n] | c <_{v_i} d\}} &>
		\frac{2mn+n}{4m}] < e^{\frac{-n}{8m^2}} \text{ and} \\
		\Pr[\norm{\{i \in [n] | c \prec_{v_i} d\}} &< \frac{3n}{4m}] <
		e^{\frac{-n}{8m^2}}
		\end{align}
		by creating random variables $X_i$ for the events $c <_{v_i} d$
		and $c \prec_{v_i} d$ and adjusting their values to utilize the
		theorem appropriately.
		Then use the union bound to prove Claim 2.
	\end{pfsketch}

	Now we can prove the main statements of the theorem.
	Item 1 follows from applying the union bound to claim 2:

	\[\bigvee_{d \in C \setminus \{c\}}\Pr[(\norm{\{i \in [n] | c <_{v_i} d\}} >
		\frac{2mn+n}{4m}) \vee
		(\norm{\{i \in [n] | c \prec_{v_i} d\}} < \frac{3n}{4m})].\]

	Item 2, follows similarly using the union bound:

	\[\bigvee_{c,d \in C ~ s.t. ~ c \neq d}\Pr[(\norm{\{i \in [n] | c <_{v_i} d\}} >
		\frac{2mn+n}{4m}) \vee
		(\norm{\{i \in [n] | c \prec_{v_i} d\}} < \frac{3n}{4m})].\]

	Note that $\norm{\{c,d \in C ~ s.t. ~ c \neq d\}} = (m^2-m)$.
\end{proof}

